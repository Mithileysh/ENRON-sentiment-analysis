\documentclass{vldb}
\usepackage{graphicx}
\usepackage{balance}

% COMANDO PER EVIDENZIARE PARTI DA LEGGERE ATTENTAMETE
\usepackage{color,comment}
\usepackage{soul}
\definecolor{lightcyan}{RGB}{210, 210, 250}
\newcommand{\hlc}[2][lightcyan]{{\sethlcolor{#1}\hl{#2}}}
%\newcommand{\hlc}{}


\begin{document}

\title{E1: ENRON Sentiment}


\numberofauthors{2} 
\author{
\alignauthor
Andrea Jemmett\\
       \affaddr{Vrije Universiteit Amsterdam}\\
       \affaddr{Amsterdam, the Netherlands}\\
       \email{andreajemmett@gmail.com}
\alignauthor
Enrico Rotundo\\
       \affaddr{Vrije Universiteit Amsterdam}\\
       \affaddr{Amsterdam, the Netherlands}\\
       \email{enrico.rotundo@gmail.com}
}


\maketitle

\begin{abstract}
Email dataset analysis is a challenging task in terms of quantity and poor-structured data.
Anyway, the availability of big computational infrastructures such as cluster computers helps to face the former issue.
Indeed, such platforms provide high and scalable computing and unload the programmer from the burden of managing most of its parallelisation and distribution.
Unfortunately, email datasets usually come as unstructured dataset in the form of text files or, whenever they contain any markup structure, the actual data might not be well formed.
In that case, the data could be human-readable but hardly parsable by a machine.
Therefore, the analysis should include additional mining steps and many integrity checks, in order to minimise any possible inconsistencies.  

In the past years, several email datasets from diverse sources have been publicly released.
In this paper, we analyse the famous ``Enron Corpus'' which contains 620k messages in about 150 mailboxes belonging to Enron employees involved in a court case.
We extract and analyse sentiments within those messages using functional programming together with a well known engine for large-scale data processing. 
Thus, the analysis is run in a high performance computing cluster.   
We present our result as an interactive visualisation of the sentiment spread via emails together with the company's stock price of the same period.
\hlc{TODO: aggiungere le conclusioni!!!!!!!!!!!}

\end{abstract}


\section{Introduction}
Email is, at least on the user side, a simple mean of communication.
Its popularity is probably due to the simplicity of usage: users can send textual messages and attachments to other addresses, also from mobile devices~\cite{chen2002enterprise}.   
Thus, in the digital era it became very a popular way of communication between privates and companies.
Normally, corporate emails are characterised by a specific structure, for example \textit{user@company.com}, where the \textit{user} suffix is a mailbox identifier and \textit{company.com} is a distinguishable company web domain.
A corporate mailbox server can handle and store thousands of inbound or outbound messages every day, collecting quite a huge amount of exchanged data.

Email dataset analysis consists in analyse a dumped data in order to extract specific information (e.g., communication patterns, sentiment analysis, etc.).
Such analysis is expensive in terms of computation: the data is often composed of a multitude of items that have to be processed individually.
Therefore, such tasks are normally run in distributed environments which allow high degrees of parallelisation.
Cluster computing provides a platform for executing complex parallel tasks in a programmer-friendly environment~\cite{buyya1999high, zaharia2010spark, shvachko2010hadoop}.
This means the programmer does not explicitly code how to parallelise the computation.
Moreover, such systems rely on distributed file systems which provide large storage capabilities and support for redundancy and distributed accesses~\cite{weil2006ceph}.

Furthermore, email datasets usually come in a semi-structured fashion in the sense that the actual data might not be well formed. 
For instance, recipients attributes and email's body can be difficult to parse.
Thus, the analysis should include some validation steps which might slow down the whole development process.

 
~\cite{shetty2004enron}


\section{Related work}
TODO

\section{Research questions}
TODO

\section{Project setup}
TODO

\section{Experiments}
TODO

\section{Conclusions}
TODO


\section{Acknowledgments}
SURFsara maybe?...

\clearpage
\balance
\bibliographystyle{abbrv}
\bibliography{bibliography} 

\begin{appendix}
TODO

\section{BLA BLA}
TODO

\end{appendix}



\end{document}
